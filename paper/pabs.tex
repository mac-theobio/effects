% Please keep the abstract below 300 words
\section*{Abstract}


In generalized linear (mixed) models that involve complex multiplicative interactions, multi-parameter variables, additional non-focal predictors or a non-linear link function, outcome plots (prediction and effect plots) can aid in understanding difficult-to-interpret coefficient estimates. Outcome plots depend on the choices we make about the non-focal predictors.
The most common approach is to generate estimates (central estimates, predictions and effects) at a reference point, usually the mean of non-focal predictor. We call this mean-based approach and it estimates effect of an average case in the population. In the presence of additional non-focal predictors, non-linear link functions, random effect terms, etc., mean-based approach generates estimates that are biased and may not be consistent with the observed quantities. An alternative is the observed-value-based approach which estimates the average effect in the population. Moreover, effect-styled confidence intervals provides an alternative and a more clear way to describe uncertainty associated with the focal predictor. In addition to theoretical and methodical comparison, using simulation, we illustrate the two approaches and show that they can produce substantially different results and that observed-value-based approach can not only produce estimates consistent with the observed values, but also appropriate for bias correction. We also present an alternative way, effect-styled confidence intervals, to describe uncertainties associated with the central estimates.

% Please keep the Author Summary between 150 and 200 words
% Use first person. PLOS ONE authors please skip this step. 
% Author Summary not valid for PLOS ONE submissions.   
% \section*{Author summary}

