% Please keep the abstract below 300 words
\section*{Abstract}

\jd{I haven't looked at the Abstract for a long time; will get back to it. Moving it now to the main document file.}

In generalized linear (mixed) models that involve complex multiplicative interactions, multi-parameter variables, additional non-focal predictors or a non-linear link function, outcome plots (prediction and effect plots) can aid in understanding difficult-to-interpret coefficient estimates. Outcome plots depend on the choices we make about the non-focal predictors.
The most common approach is to generate estimates (central estimates, predictions and effects) at a reference point, usually the mean of non-focal predictor. We call this mean-based approach and it estimates effect of an average case in the population. In the presence of additional non-focal predictors, non-linear link functions, random effect terms, etc., mean-based approach generates estimates that are biased and may not be consistent with the observed quantities. An alternative is the observed-value-based approach which estimates the average effect in the population. Moreover, effect-styled confidence intervals provides an alternative and a more clear way to describe uncertainty associated with the focal predictor. In addition to theoretical and methodical comparison, using simulation, we illustrate the two approaches and show that they can produce substantially different results and that observed-value-based approach can not only produce estimates consistent with the observed values, but also appropriate for bias correction. We also present an alternative way, effect-styled confidence intervals, to describe uncertainties associated with the central estimates.

\BC{LET US DISCUSS: What do we want to focus on? common way of generating the plots (input variables) vs model center, prediction- vs effect-style or bias correction, or everything?}
\jd{Certainly the first two. Do you think bias should be a separate paper?}

\section{Introduction}

Plots of predicted values of an outcome against predictors (often called effect plots or prediction plots, generalized here as “outcome plots”) are often a useful way to summarize the results of a statistical model. These can be used to illustrate the shape of model response and associated uncertainties, and may be easier to interpret than tables of coefficient estimates \citep{brambor_understanding_2006, berry_improving_2012, leeper2017interpreting}. 

\begin{gloss}
\subsection*{Glossary}
\begin{description}
\item [Input variables] Observed (scientific) variables underlying an inference or exploration. Input variables for our simulated ecological data set are N, P and K, plus the biomass used as a response variable.

\item [Focal predictor] The input variable on the x-axis of an outcome plot.

\item [Linear predictor variables] The variables which are combined to make the linear predictor (corresponding to the columns in the model matrix). Each input variable contributes to one or more linear predictor variables. N, P and K in our ecological data set are all linear predictor variables; so are combinations like P:K or N$\mbox{}^2$ when they appear in the model.

\item[Multi-parameter variables (MPVs)] Input variables which correspond to more than one linear predictor variable -- for example, variables with more than two categories, or variables with spline or polynomial response. 

\item [Model center] The point corresponding to the mean of the linear predictor variables. In an ordinary linear model, the central estimate at this point will be the mean of the response variable. 

\item [Reference point] The values (or sets of values) chosen for non-focal predictors when estimating the predictions and effects. Can be the model center, chosen baseline values, or a weighted or unweighted mean across categories. We will also discuss averaging over a set of quantiles or observations as a reference point.

\item [Anchor] The values chosen for variables related the focal predictor when estimating effect-style confidence intervals. The anchor choice does not affect the central estimates, nor prediction-style confidence intervals. Often but not always based on the model center. 

\end{description}
\end{gloss}

To make an outcome plot, we use a \emph{focal} predictor on the x-axis and values of the outcome variable on the y-axis. The resulting plot will depend on choices about other (non-focal) predictors. In an \emph{ordinary} linear model, the non-focal choices may have a simple additive effect on the central estimate. However, when the focal predictor has interactions or non-linear response functions (e.g., spline or polynomial), non-focal choices can also affect the shape of the estimated curve. 
In a \emph{generalized} linear model, there is usually a non-linear link function, which complicates the picture of additivity.
Additional challenges arise when dealing with “mixed” models, which incorporate random effects.

We endeavor here to make a conceptual distinction between two functions of outcome plots: prediction and effects. 
The primary difference lies in how we describe the uncertainties around the central estimate. 
If our goal is to \emph{predict} what we learn about the outcome variable by measuring the focal predictor, then we may want to capture a variety of sources of uncertainty, including that due to the intercept, focal and non-focal predictors, and random effects.
Conversely, if we wish to focus on the \emph{effect} of a focal predictor only, we might want to isolate uncertainty due to coefficients associated with that predictor.
If we follow this convention, we expect effects plots to generally have narrower confidence intervals (CIs) than prediction plots, since less uncertainty is taken into account.

The other distinction relates not to the method of calculating CIs, but to the model chosen for the plot. 
In general, if we want to predict based on a focal parameter, we are likely to prefer a univariate model that only contains terms related to that predictor, thus addressing the question: if we know (only) the focal parameter, what can we predict about the outcome?
If we want to know the effects of a predictor, we may want to control for covariates with a multivariate model, in order to estimate “direct” effects; leave covariates out, in order to estimate “total” effects (direct plus indirect); or take an intermediate strategy.
These address different versions of the question: if we change the focal parameter (possibly allowing it also to affect other covariates), what is the expected effect on the outcome variable (see, e.g., \citep{shi_evidence_2017}).

We discuss the idea of a model “center” and show how it simplifies thinking about outcome plot choices, particularly in the case of ordinary linear models. We also discuss biases that arise from non-linear link functions in generalized linear models, and how such biases can be addressed.

%% \clearpage

\section{Ordinary linear models}

\begin{figure}
\begin{center}
\includegraphics{eco.splash.Rout.pdf}
\end{center}
\caption{Outcome plots with effect- (dotted straight lines) and prediction-style (dashed curved lines) \CIs. The panel on the left anchors the effect-style intervals at the model center; the panel on the right has a specific anchor chosen for biological reasons. Anchor point does not affect prediction-style confidence intervals. Points show the fitted data; the large grey point shows the mean values of predictor and the response.}
\flab{splash}
\end{figure}

\fref{splash} shows effect- and prediction-style confidence intervals for a hypothetical data set. Prediction-style confidence intervals do not depend on an anchor choice, and correspond to \CIs\ for the predicted mean value of Biomass for the given value of Nitrogen (note that these differ from the broader \emph{prediction intervals} which give \CIs\ for observations rather than the predicted means, and which are not discussed further here). Effect-style confidence intervals show the range of slopes corresponding to the model's \CIs\ for the uncertainty due specifically to the effect of the focal predictor. In this simple case, effect-style CIs, unlike prediction-style CIs, have a clear visual correspondence with statistical clarity: both CI lines slope in the same direction (both up, or both down) exactly when the P value for the effect of the predictor is below the threshold for statistical clarity.

\subsection{Direct and indirect effects}

In a model with more than one predictor, we may be interested in an estimate of the \emph{direct} effects of one predictor (after controlling for others). If predictors are correlated, the direct effect of a predictor may not reflect the visible relationship between the predictor and the outcome variable.

\begin{figure}
\begin{center}
\includegraphics{eco.uniMulti.Rout.pdf}
%%
\caption{Outcome plots with effect- (dotted straight lines) and prediction-style (dashed curved lines). The panel on the left shows the results of a univariate model fit; the panel on the right shows a multivariate fit (and effect-style lines only, see text).}
\jd{How about now (see text also)?}
%%
\flab{uniMulti}
\end{center}
\end{figure}
%%

\fref{uniMulti} shows an example. The predicted \emph{overall} relationship between nitrogen and biomass is shown in the left panel. This reflects not only direct and indirect effects of nitrogen on biomass, but also any correlations that may be driven by other causes. We may be interested in an effect-style plot here, if we think nitrogen is the main driver of observed effects and we want to estimate total (direct plus indirect) effects. Or we may be interested in a prediction-style plot of the expected value of biomass for a given \emph{observed} value of nitrogen.

The predicted \emph{direct} relationship between nitrogen and biomass is shown in the right panel. This does not track the visible relationship in this case, but instead reflects the modeled \emph{direct} effect of nitrogen. The meaning of such a plot will depend on which variables we choose to control for. Prediction-style plots are less likely to be useful for multivariate models, so here we show only effect-style intervals.

\subsection{Reference points}

To make an outcome plot, we vary the value of the focal predictor, but we need to choose values for other predictors -- the reference point for the plot. Our default is to do this using the \emph{model center}. We define the model center as the average value of all of the “linear predictor variables” which go into the model matrix. This is not, in general, the same as averaging over the “input variables” which are measured and provided as data (see Glossary box). 

The model center does not always correspond to a physically possible data point. For example, the mean of an interaction term across the data is generally not the same as the value of the same term evaluated when each of its components is at its mean. The predicted outcome at the model center, however, is always the same as the mean predicted outcome (in a generalized linear model, this is true on the link scale where the model operates, not the original scale of the outcome variable). For this reason, using the model center tends to make outcome plots match the data better than other approaches, including the most common approach of taking the mean of input variables only.

\begin{figure}
\begin{center}
\includegraphics{eco.multiPK.defAnchor.centers.plot.Rout.pdf}
\end{center}
\caption{Outcome plots for a model with a non-focal interaction. Black lines show an effect plot (mean estimate and CIs) made using the model center (average of linear predictor variables) as a reference point. Blue lines show the same approach made by centering based on input variables; due to the interaction the central prediction line does not go through the center point.}
\flab{centers}
\end{figure}

\fref{centers} shows an example, based on simulated data with an interaction between correlated variables. The two approaches correctly show the same information about effects of the focal variable, but the one using the model center as a reference point (the \pkg{varpred} default) is centered with respect to the data, while the one that centers based on input variables alone (the most common default, including \pkg{emmeans} and \pkg{effects}) is offset vertically. This is due to the interaction term not being centered when the input variables are centered; these two approaches give identical results for the simpler model in \fref{splash}.

\subsection{Multi-parameter predictors}

A single input variable can be associated with multiple parameters either by being involved in an interaction, as above, or when a non-linear response (typically spline or polynomial) is specified. In this case, the anchor point can be chosen separately for each of the linear predictor variables corresponding to the non-linear response. Again, our recommended method is to use the model center. The natural alternative is to pick a value of the input variable (either a scientifically meaningful value, or the mean).

\begin{figure}
\begin{center}
\includegraphics{eco.quadComp.Rout.pdf}
\end{center}
\caption{Outcome plots for a model with a non-linear response to the focal variable. The left plot is anchored at the model center, which does not correspond to a possible value for the focal input variable (nitrogen). The right-hand plot is anchored at the mean value of the input variable (nitrogen). }
\flab{quadComp}
\end{figure}

An example is shown in \fref{quadComp}. We fit a univariate, \emph{quadratic} response to nitrogen for the biomass. Both panels show the same model, and have the same response curves and prediction CIs. The panel on the left shows the default effect CIs, anchored at the model center.
Because we average separately over the two different linear predictor variables, 
the model center does not correspond to a real value of nitrogen and is therefore not seen on the plot. 
Thus, no anchor point where the effect-style CIs come together is visible. 
The panel on the right is anchored at a real nitrogen value (in particular, the mean across the sampled values), and we see the effect-style CIs coming together at this point.

In neither case is the picture as clear to interpret as the simplest case (\fref{splash}). We argue though that the picture on the left is a good way to show the shape of the model fit, while focusing on the uncertainty due to the parameter values due to the focal variable alone. The physically interpretable anchor on the right, even though chosen to correspond to the mean value of nitrogen, is fair from the model center and generates effect-style CIs that are at some points wider than the prediction-style CIs.

\section{Generalized linear models}

\section{Discussion}

Generalized linear (mixed) models are widely used in various fields, including public health. In a model involving difficult-to-interpret coefficient estimates, an outcome plot can aid in understanding and summarizing the results. In particular, a prediction plot would be appropriate if the goal is to capture every uncertainty in the model for a particular focal predictor or if we are interested in total effect. Conversely, an effect plot is preferable if we want to focus on the uncertainty associated with a focal predictor only or if we are interested in the direct effect.

The mean-based approach is widely used to create outcome plots. However, in a model with complex interaction, MPVs or categorical variables, it is sensitive to the choice of the reference point. We have demonstrated that a model-center-based reference point is generally a stable choice and provides estimates more consistent with the observed quantities as compared to common input-variable mean-based approach.

In a model with a non-linear link function such as a logistic or exponential function, the generated central estimate curve may not match well with the observed data, i.e., our description for bias. In such a model, the observed-value-based approach provides a way to generate more consistent estimates and is preferable to the widely used mean-based approach.

The argument and results we present in this paper support a greater need for a shift in focus on how to summarize these kinds of models. From our theoretical, methodological and simulation results, researchers using these models should, in the absence of theoretical justification, report predictions based on the observed-value approach or at least attempt to compare the two approaches before settling on the most appropriate in answering their research question. Moreover, we provide \proglang{R} package, \pkg{vareffects}, which implements these methods and is available on GitHub (\href{https://github.com/mac-theobio/effects}{https://github.com/mac-theobio/effects}).

Our simulation examples focused on simple linear and logistic models due to their wide range of usage and application. These models also act as a starting point for building other complex models, including mixed effect models and models with categorical predictors. The logic for extending to more complex models, including other forms of non-linear link functions, is straightforward. The components needed for extension are the correct linear predictor and the inverse link function; everything else generalizes. In addition, our \proglang{R} package implementation already extends to and supports most of the non-linear link functions and mixed model framework, including multivariate binary outcome models.

\section{Supporting information}

% Include only the SI item label in the paragraph heading. Use the \nameref{label} command to cite SI items in the text.
\paragraph*{S1 Appendix.}
\label{S1_Appendix}
{\bf Cubic polynomial interaction simulation.} Consider a hypothetical simulation which simulates household size as a function of household wealth index and cubic function of the age of the household head, specified as follows:
%
\begin{align}\label{sim:lm_cubic}
\mathrm{hh~size}_i &= \beta_0 + \beta_{\mathrm{A_1}}\mathrm{Age}_i + \beta_{\mathrm{A_2}}\mathrm{Age}^2_i + \beta_{\mathrm{A_3}}\mathrm{Age}^3_i + \beta_{\mathrm{W}}\mathrm{Wealthindex}_i + \epsilon_i \nonumber\\
\mathrm{Age}_i &\sim \mathrm{Normal}(0, 1) \nonumber\\
\mathrm{Wealthindex}_i &\sim \mathrm{Normal}(0, 1) \nonumber\\
\epsilon_i &\sim \mathrm{Normal}(0, 10) \nonumber\\
\beta_0 &= 20 \nonumber\\
\beta_{\mathrm{A}_1} &= 0.1 \nonumber\\
\beta_{\mathrm{A}_2} &= 0.8 \nonumber\\
\beta_{\mathrm{A}_3} &= 0.3 \nonumber\\
\beta_{\mathrm{W}} &= -0.5 \nonumber\\
i &= 1,\cdots, 100
\end{align}


\paragraph*{S2 Appendix.}
\label{S2_Appendix}
{\bf Binary outcome simulation.} Consider a simple simulation for improved water quality in Nairobi slums, such that the status is $1$ for improved and $0$ for unimproved water quality. In addition to the focal predictor, age of the household head, we add wealth index. In particular:
%
\begin{align}\label{sim:glm_two_pred}
\mathrm{status}_i &\sim \mathrm{Bern}(\mathrm{P_i}) \nonumber\\
\mathrm{logit}(\mathrm{P_i}) &= \eta_i \nonumber\\
\mathrm{\eta}_i &= \beta_0 + \beta_{\mathrm{A}}\mathrm{Age}_i + \beta_{\mathrm{W}}\mathrm{Wealthindex}_i \nonumber\\
\mathrm{Age}_i &\sim \mathrm{Normal}(0, 1) \nonumber\\
\mathrm{Wealthindex}_i &\sim \mathrm{Normal}(0, 1) \nonumber\\
\beta_0 &= 5 \nonumber\\
\beta_{\mathrm{A}} &= 0.5 \nonumber\\
\beta_{\mathrm{W}} &= 1.5 \nonumber\\
i &= 1,\cdots, 10000
\end{align}

\paragraph*{S3 Appendix.}
\label{S3_Appendix}
{\bf Mediated effect simulation.} Next, we consider a simple indirect mediation previously described and simulate a binary outcome model such that:

\begin{align}\label{sim:simple_mediate}
%% z_i &\sim \mathrm{Bern}(\mathrm{P_i}) \nonumber\\
%% \mathrm{logit}(\mathrm{P_i}) &= \eta_i \nonumber\\
z_i &= \beta_0 + \beta_{xz} x_i + \beta_{yz} y_i \nonumber\\
y_i &= \rho x_i + \sqrt{1-\rho^2} y_y \nonumber\\
x_i &\sim \mathrm{Normal(0, 1)} \nonumber\\
y_y &\sim \mathrm{Normal(0, 1)} \nonumber\\
\rho &= 0.8 \nonumber\\
\beta_0 &= 5 \nonumber\\
\beta_{xz} &= 0.2 \nonumber\\
\beta_{yz} &= 1.5 \nonumber\\
i &= 1,\cdots, 10000
\end{align}


\section*{Acknowledgments}

This work was supported by a grant to Jonathan Dushoff from the Natural Sciences and Engineering Research Council of Canada (NSERC) Discovery.

\section*{Author Contributions}

\textbf{Conceptualization:} Jonathan Dushoff, Steve Cygu

\noindent\textbf{Software:} Steve Cygu, Benjamin M. Bolker

\noindent\textbf{Writing – original draft:} Steve Cygu

\nolinenumbers

