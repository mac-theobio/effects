% Title must be 250 characters or less.
\begin{flushleft}
{\Large
\textbf\newline{Describing the curves: uncertainty estimation and bias correction for predictor effects in simple and generalized linear (mixed) models} % Please use "sentence case" for title and headings (capitalize only the first word in a title (or heading), the first word in a subtitle (or subheading), and any proper nouns).
}
\newline
% Insert author names, affiliations and corresponding author email (do not include titles, positions, or degrees).
\\
Steve Cygu\textsuperscript{1},
Benjamin M. Bolker\textsuperscript{1,2},
Jonathan Dushoff\textsuperscript{1,2}
\\
\bigskip
\textbf{1} School of Computational Science and Engineering, McMaster University, Hamilton, Ontario, Canada
\\
\textbf{2} Department of Biology, McMaster University, Hamilton, Ontario, Canada
\\
\bigskip


% Use the asterisk to denote corresponding authorship and provide email address in note below.
* cygu@aims.ac.za

\end{flushleft}
% Please keep the abstract below 300 words
\section*{Abstract}

In applications that use generalized linear (mixed) models, outcome predictions are often of interest. For models that involve complex multiplicative interactions, additional non-focal predictors or nonlinear link functions, the estimated coefficients are not readily interpretable. A general way to summarize these kind of models is through predictions, that depend not only on choice of representative values of focal predictor but are also sensitive to which values of non-focal predictors are chosen. The most common approach is generating predictions at a reference point, usually the mean, of non-focal predictor, i.e., mean-based (centered) which could be considered as the effect of an average case in the population. In the presence of additional non-focal predictors, nonlinear link functions, random effect terms, etc., mean-based approach generate predictions that are biased and may not be consistent with the observed quantities. An alternative is the whole-sample-based approach which estimates the average effect in the population. Moreover, anchored effects provide an alternative and a more clear way to describe uncertainty associated with the focal predictor of interest. In addition to theoretical and methodical comparison, using simulation, we illustrate the two approaches and show that they can produce substantially different results and that whole-sample-based approach can not only produce estimates consistent with the observed values, but also appropriate for bias correction. We also present an alternative way, anchored confidence intervals, to describe uncertainties associated with these predictions.

\section*{Definitions}

\begin{itemize}
\item \textbf{Input variables}: Scientific variables underlying an inference or exploration. The focal predictor we use for effects or predictions is an input variable.
\item \textbf{Model variables}: Variables that represent columns in the model matrix. Each input variable will correspond to one or more model variables. In particular, variables with more than two categories, or variables modeled with a spline or polynomial response, will correspond to more than one model variables.
\item\textbf{Model center:} A point corresponding to a column-wise mean of the model matrix (the mean of one or more model variables). The center point for a set of model variables corresponding to an input variable may not represent a possible value of the input variable.
\item \textbf{Reference point}: Value or values chosen for \emph{non-focal predictors}, when estimating effects. Typically the center point, but can instead be a population of quantiles or observations. 
\item \textbf{Anchor}: The value chosen for the \emph{focal predictor} when estimating effect confidence intervals (anchor choice does not affect the estimate). Typically chosen as the center point of the model variables corresponding to the focal input variable.
\end{itemize}

% Please keep the Author Summary between 150 and 200 words
% Use first person. PLOS ONE authors please skip this step. 
% Author Summary not valid for PLOS ONE submissions.   
\section*{Author summary}

\fix
