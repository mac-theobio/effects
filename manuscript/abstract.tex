% Title must be 250 characters or less.
\begin{flushleft}
{\Large
\textbf\newline{Outcome plots: uncertainty estimation and bias correction for predictions and effects in simple and generalized linear (mixed) models} % Please use "sentence case" for title and headings (capitalize only the first word in a title (or heading), the first word in a subtitle (or subheading), and any proper nouns).
}
\newline
% Insert author names, affiliations and corresponding author email (do not include titles, positions, or degrees).
\\
Steve Cygu\textsuperscript{1, *},
Benjamin M. Bolker\textsuperscript{1,2},
Jonathan Dushoff\textsuperscript{1,2}
\\
\bigskip
\textsuperscript{1}School of Computational Science and Engineering, McMaster University, Hamilton, Ontario, Canada
\\
\textsuperscript{2}Department of Biology, McMaster University, Hamilton, Ontario, Canada
\\
\bigskip


% Use the asterisk to denote corresponding authorship and provide email address in note below.
* cygu@aims.ac.za

\end{flushleft}
% Please keep the abstract below 300 words
\section*{Abstract}

\jd{This talks about a lot of terms that we don't define. Remember to rewrite!}
\fix{BC to rewrite}

In applications that use generalized linear (mixed) models, outcome predictions are often of interest. For models that involve complex multiplicative interactions, additional non-focal input variables or nonlinear link functions, the estimated coefficients are not readily interpretable. A general way to summarize these kind of models is through predictions, that depend not only on choice of representative values of focal input variables but are also sensitive to which values of non-focal input variables are chosen. The most common approach is generating predictions at a reference point, usually the mean of non-focal predictor. We call this mean-based approach and it estimates effect of an average case in the population. In the presence of additional non-focal input variables, nonlinear link functions, random effect terms, etc., mean-based approach generates predictions that are biased and may not be consistent with the observed quantities. An alternative is the whole-sample-based approach which estimates the average effect in the population. Moreover, effects confidence intervals provide an alternative and a more clear way to describe uncertainty associated with the focal input variable. In addition to theoretical and methodical comparison, using simulation, we illustrate the two approaches and show that they can produce substantially different results and that whole-sample-based approach can not only produce estimates consistent with the observed values, but also appropriate for bias correction. We also present an alternative way, effects confidence intervals, to describe uncertainties associated with these predictions.

% Please keep the Author Summary between 150 and 200 words
% Use first person. PLOS ONE authors please skip this step. 
% Author Summary not valid for PLOS ONE submissions.   
% \section*{Author summary}

